\documentclass[a4paper,12pt]{article}

\usepackage[french,italian]{babel}
\usepackage[T1]{fontenc}
\usepackage[utf8]{inputenc}
\frenchspacing 
\title{Appunti Algoritmi e Strutture Dati}
\author{Luca Seggiani}
\date{20 Marzo 2024}

\usepackage{forest}

\begin{document}
\maketitle
\section{Alberi binari}
Un'albero binario è definito come segue:
\begin{itemize}
  \item L'albero vuoto è un'albero binario.
  \item Un nodo $p$ più due alberi binari $B_s$ e $B_d$ formano un'albero binario.
\end{itemize}

\begin{center}
\begin{forest}
  [$p$ [$B_s$] [$B_d$]]
\end{forest}
\end{center}

a loro volta $B_s$ e $B_d$ saranno alberi binari, quindi:

\begin{center}
\begin{forest}
  [$p$ [$B_s$ [$B'_s$][$B'_d$]] [$B_d$[$B'_s$][$B''_d$]]]
\end{forest}
\\...
\end{center}

a questo punto diremo che:
\begin{itemize}
  \item $p$ è la \textbf{radice} dell'albero;
  \item $B_s$ è il \textbf{sottoalbero sinistro} di $p$;
  \item $B_d$ è il \textbf{sottoalbero destro} di $p$;
  \item gli alberi sono etichettati.
\end{itemize}

diciamo inoltre che ogni albero senza sottoalberi (o figli) è una foglia. Un nodo sarà padre rispetto al
loro figlio, i nodi precedenti saranno glli antecedenti e quelli successivi i discendenti.
Il livello di un nodo è il numero dei suoi antecedenti, mentre il livello di un'albero è il livello massimo
dei suoi nodi.

\end{document}
