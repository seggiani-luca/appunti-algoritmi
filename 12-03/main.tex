\documentclass[a4paper,12pt]{article}

\usepackage[french,italian]{babel}
\usepackage[T1]{fontenc}
\usepackage[utf8]{inputenc}
\frenchspacing 
\title{Appunti Algoritmi e Strutture Dati}
\author{Luca Seggiani}
\date{12 Marzo 2024}

\usepackage{listings}
\usepackage{xcolor}
\usepackage{hyperref}

\definecolor{codegreen}{rgb}{0,0.6,0}
\definecolor{codegray}{rgb}{0.5,0.5,0.5}
\definecolor{codepurple}{rgb}{0.58,0,0.82}
\definecolor{backcolour}{rgb}{0.95,0.95,0.92}

\lstdefinestyle{code-style}{
    backgroundcolor=\color{backcolour},   
    commentstyle=\color{codegreen},
    keywordstyle=\color{magenta},
    numberstyle=\tiny\color{codegray},
    stringstyle=\color{codepurple},
    basicstyle=\ttfamily\footnotesize,
    breakatwhitespace=false,         
    breaklines=true,                 
    captionpos=b,                    
    keepspaces=true,                 
    numbers=left,                    
    numbersep=5pt,                  
    showspaces=false,                
    showstringspaces=false,
    showtabs=false,                  
    tabsize=2
}

\lstset{style=code-style}

\begin{document}
\maketitle
\section{Insertion sort}
L'insertion sort utilizza il metodo più intuitivo per ordinare un vettore: si prende ad ogni iterazione 
il primo elemento, si confronta con ogni suo precedente finchè non si trova la sua posizione corretta, e poi
si spostano tutti i precedenti necessari di una posizione in avanti per lasciargli posso. Notiamo che nell'implementazione
in pseudocodice il while usato per spostare gli elementi ha la conseguenza di lasciare il primo elemento con un 
valore duplicato: usiamo allora la variabile ausiliaria current per rimettere il valore giusto al suo posto.

\begin{lstlisting}[language=C++]
void sort(int* vett, int dim) {
  int current = 0;
  int p = 0;
  for(int i = 1; i < dim; i++) {
    current = vett[i];
    p = i - 1;

    while(p >= 0 && vett[p] > current) {
      vett[p + 1] = vett[p];
      p--;
    }
    vett[p + 1] = current;
  }
}
\end{lstlisting}

Possiamo fare l'analisi della complessità e trovare che, come per tutti gli algoritmi di ordinamento in-place, la complessità
massima è di $O(n^2)$. Il worst case sarà chiaramente quello di un vettore ordinato al contrario, cioe in ordine decrescente.

\section{Debugging}
Il processo di debugging consiste nel rimuovere eventuali errori dal codice scritto. Si possono distinguere le seguenti
categorie di debugging:

\begin{itemize}
  \item Visuale: osservando lo stato del programma in un qualsiasi momento (ad esempio scrivendo sul buffer di uscita
    le informazioni che ci interessano);
  \item Debugger: utilizzando appositi software chiamati debugger (GDB, DDD) che forniscono alcuni strumenti utili al debugging
    quali i breakpoint, l'analisi della memoria in tempo reale, ecc...;
  \item Compilatore: utilizzando i messaggi di errore forniti dallo stesso compilatore;
  \item Analisi della memoria: utilizzando software come Valgrind per gestire correttamente la memoria ed individuare
    leak e segmentazioni.
\end{itemize}

\section{Programmi in memoria dinamica}
Utilizziamo la programmazione in memoria dinamica nel caso in cui le dimensioni d'istanza del problema che ci interessa
non siano note a tempo di compilazione. Diciamo ad esempio di voler immagazinare una serie di n numeri. Potremmo
voler usare una lista, ma a quel punto dovremmo accettare di dover richiedere tutte le nostre letture sulla struttura dati
in tempo $n$. Utilizzando invece un vettore dinamico, avremmo una complessità di $O(1)$ per ogni accesso. Tutte queste strutture
dati possono essere trovate nell'implementazione della libreria standard, ovvero la:
\par\smallskip
\textbf{Standard Template Library (STL)} \\
La libreria STL definisce una serie di container, ovvero strutture dati predefinite che possono venire utilizzate
nel nostro codice. Vediamo per esempio un vettore definito attraverso la STL:
\begin{lstlisting}[language=C++]
vector<int> stlArray
\end{lstlisting}
qui vector indica il tipo di container, lal voce <int> definisce il tipo di dati immagazinati nel container, 
e stlArray il nome del vettore. Per aggiungere un elemento al vettore, potremo ad'esempio;
\begin{lstlisting}[language=C++]
stlArray.push_back(val);
\end{lstlisting}
chiamando la funzione membro push\_back sulla stlArray (altro non è che un'istanza di classe) con argomento val.
Per leggere un valore dal vettore potremmo usare la ridefinizione dell'operatore []:
\begin{lstlisting}[language=C++]
stlArray[index];
\end{lstlisting}
potremo ottenere iteratori sulla testa e sulla coda del vettore con:
\begin{lstlisting}[language=C++]
stlArray.begin();
stlArray.end();
\end{lstlisting}
ed ottenerne la dimensione con:
\begin{lstlisting}[language=C++]
stlArray.size();
\end{lstlisting}

Riguardo agli algoritmi di ordinamento visti finora, la libreria STL fornisce una funzione:
\begin{lstlisting}[language=C++]
sort(stlArray.begin(), stlArray.end());
\end{lstlisting}
dove i due argomenti sono iteratori generici tra i quali verrà effettuato l'ordinamento. L'algoritmo
dell'ordinamento è implementation-dependant, cioò dipende dall'implementazione usata, ma assicura 
sempre una complessità di $O(n\log{n})$.
\par\smallskip
Ulteriori informazioni su qualsiasi componente della STL si può trovare su siti come: \\
\url{https://cplusplus.com/reference} \\

\end{document}
