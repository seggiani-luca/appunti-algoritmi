\documentclass[a4paper,12pt]{article}

\usepackage[french,italian]{babel}
\usepackage[T1]{fontenc}
\usepackage[utf8]{inputenc}
\frenchspacing 
\title{Appunti Algoritmi e Strutture Dati}
\author{Luca Seggiani}
\date{27-28 Febbraio 2024}
\usepackage{listings}
\usepackage{xcolor}

\definecolor{codegreen}{rgb}{0,0.6,0}
\definecolor{codegray}{rgb}{0.5,0.5,0.5}
\definecolor{codepurple}{rgb}{0.58,0,0.82}
\definecolor{backcolour}{rgb}{0.95,0.95,0.92}

\lstdefinestyle{code-style}{
    backgroundcolor=\color{backcolour},   
    commentstyle=\color{codegreen},
    keywordstyle=\color{magenta},
    numberstyle=\tiny\color{codegray},
    stringstyle=\color{codepurple},
    basicstyle=\ttfamily\footnotesize,
    breakatwhitespace=false,         
    breaklines=true,                 
    captionpos=b,                    
    keepspaces=true,                 
    numbers=left,                    
    numbersep=5pt,                  
    showspaces=false,                
    showstringspaces=false,
    showtabs=false,                  
    tabsize=2
}

\lstset{style=code-style}

\begin{document}
\maketitle
\section{Introduzione agli algoritmi}
Un'algoritmo non è altro che un procedimento che descrive una sequenza di passi ben definiti atti a risolvere
un problema, o più schematicamente:
\begin{itemize}
  \item Insieme finito di istruzioni
  \item Accetta un input e restituisce un output
  \item Ogni istruzione deve essere:
    \begin{itemize}
      \item ben definita
      \item eseguibile in tempo finito da un'agente di calcolo
    \end{itemize}
  \item Può eventualmente usufruire di una memoria per stadi intermedi.
\end{itemize}

noi tratteremo di algoritmi aritmetici. Alcuni esempi di tali algoritmi possono essere;
\begin{itemize}
  \item L'algoritmo di Euclide per il calcolo del MCD di due numeri
    \item Il setaccio di Eratostene per il calcolo dei primi numeri primi
\end{itemize}

Si può dire che l'algoritmo è l'essenza computazionale di un dato programma, notando però che: \\
\begin{center}
Algoritmo $\neq$ Programma
\end{center}
in quanto l'algoritmo è slegato dal linguaggio di implementazione, la macchina e l'ambiente su cui gira, mentre
il programma equivale al suo codice ormai completamente definito.

\section{Esempi basilari di algoritmi}
Si riportano due semplici algoritmi ed alcune loro caratteristiche e possibili ottimizzazioni.
\par\medskip

\textbf{Algoritmo di Euclide} \\
Nella sua definizione più semplice, l'algoritmo di Euclide si basa sul teorema:
\textit{"Il MCD fra 2 numeri è il MCD del più piccolo fra i due e la loro differenza"}
e si implementa, in pseudocodice C++, nella forma:
\begin{lstlisting}[language=C++]
int gcd(int a, int b) {
  while(a != b) {
    if(a > b) a -= b;
    else b -= a;
  }
  return a;
}
\end{lstlisting}

una chiara ottimizzazione possibile è il passaggio dall'operazione di sottrazione a quella di sottrazione ripetuta
(divisione), nel seguente modo:

\begin{lstlisting}[language=C++]
int betterGcd(int a, int b) {
  while(b > 0) {
    int temp = b;
    b = a % b;
    a = temp;
  }
  return a;
}
\end{lstlisting}

si nota che la complessità dell'algoritmo è di $O(n)$.
\par\medskip
\textbf{Setaccio di Eratostene} \\
Il setaccio di Eratostene è un algoritmo che ci permette di trovare tutti i numeri primi fra un insieme dei
primi $n$ naturali. Un approccio naive potrebbe essere quello di dividere ogni numero per ogni suo predecessore,
ottenendo una complessità nel caso peggiore di $O(n^2)$. Un'alternativa più efficiente di questo algoritmo
potrebbe considerare una tabella con i primi $n$ naturali, e successivamente cancellare ogni multiplo di ogni 
numero primo trovato finora. raggiunto un numero il cui quadrato è maggiore di $n$, l'algoritmo si arresta e
i numeri rimasti sono i primi cercati. In pseudocodice:
\begin{lstlisting}[language=C++]
void sieve(int n) {
  bool num[n];

  //preparazione array
  num[0] = false; num[1] = true;
  for(int i = 2; i < n; i++) {
    num[i] = true;
  }
  
  int i = 2;
  while(i*i < n) {
    while(num[i] == false) i++;
    
    int m = 2;
    while(i * m < n) {
      num[i * m] = false;
      m++;
    }
    i++;
  }

  for(int i = 0; i < n; i++) {
    if(num[i] == true) {
      cout << i << " ";
    }
  }
}
\end{lstlisting}

che ottiene una complessità temporale di $O(n\log{\log{n}})$.

\section{Nozioni sugli algoritmi}
Definito un determinato problema, si può utilizzare un'algoritmo su una sua certa istanza, analizzando
quindi le seguenti caratteristiche:

\begin{itemize}
  \item Problema:
    il problema da risolvere (e.g. trovare i primi $n$ primi)
  \item Dimensione dell'istanza:
    la dimensione della mole di dati su cui dovrò lavorare nell'istanza attuale (e.g. il valore di n)
  \item Modello di calcolo:
    un modello che associa ad ogni operazione effettuata dall'algoritmo un certo costo per il mio ambiente
  \item Correttezza:
    se il mio algoritmo ottiene effettivamente la risposta giusta!
\end{itemize}

oltre alle caratteristiche riportate, è poi fondamentale la valutazione della

\section{Complessità}
La complessità di un'algoritmo è una funzione che associa alla dimensione del problema il costo della sua risoluzione,
sia esso in termini di memoria o tempo (in questo caso, \textit{complessità temporale}). La dimensione dell'istanza
dipende dai suoi stessi dati, ed è in generale opportuno determinare, su una istanza generica:
\begin{itemize}
  \item Caso peggiore (best-case):
    scenario dove l'algoritmo ottiene il costo di esecuzione maggiore
  \item Caso migliore (worst-case):
    scenario dove l'algoritmo ottiene il costo di esecuzione minore
  \item Caso medio (average-case):
    scenario medio di esecuzione dell'algoritmo, necessita di un'analisi statistica dei dati delle istanze in
    entrata.
\end{itemize}

Si ricorda che l'efficienza di un'algoritmo dipende solo dal modello di calcolo adottato, ed è slegata dall'efficienza
di un qualsiasi ambiente di esecuzione.

\par\smallskip

Posso ad esempio definire, per un determinato algoritmo $P$, la funzione complessità:
$$ T_P(n) $$
che determina la complessità temporale di $P$ al variare della dimensione d'istanza $n$. Prendiamo in esempio
un semplice algoritmo che cerca il massimo di un vettore:

\begin{lstlisting}[language=C++]
int max(int vett[], int n) {
  int m = a[0];
  for(int i = 0; i < n; i++) {
    if(m > a[i]) m = a[i];
  }
  return m;
}
\end{lstlisting}

possiamo adesso definire un modello di calcolo molto semplice, che associa il costo di 1 ad ogni operazione
di assegnamento o confronto. Tenendo conto dell'assegnamento iniziale, quello dell'istruzione di ritorno, e gli 
assegnamenti e confronti eseguiti per l'inizializzazione e l'esecuzione del ciclo for, otteniamo:
$$ 1 + 1 + 4(n - 1) + 1 + 1 = 4n $$
ovvero la nostra funzione complessità dell'algoritmo. Per poter trattare di complessità senza però perdersi
nelle specifiche, visto che l'unico fattore di interesse è la scalabilità dell'algoritmo, occorre definire la

\section{Notazione O grande}
Definiamo la notazione $O(f(n))$ nel seguente modo:
$$ f(n) \in O(g(n)) \Leftrightarrow \exists n_0, \quad c > 0 \quad \mathrm{t.c.} \quad \forall n > n_0 : f(n) \leq cg(n) $$ 
si definisce poi una serie di regole per l'utilizzo di tale notazione:
\begin{itemize}
  \item fattori costanti:
    ammesso una certa costante $k$, si ha:
    $$ O(f(n)) = O(kf(n)) $$
  \item somma:
    $$ f(n) \in O(g(n)) \Rightarrow f(n) + g(n) \in O(g(n)) $$
  \item prodotto:
    $$f(n) \in O(f_1(n)), \quad g(n) \in O(g_1(n)) \Rightarrow f(n)g(n) \in O(f_1(n)g_1(n)) $$
  \item transitività:
    $$ f(n) \in O(g(n)) \land g(n) \in O(h(n)) \Rightarrow f(n) \in O(h(n)) $$
  \item costanti:
    $$ \forall k, \quad k \in O(1) $$
  \item potenze:
    $$ m \leq p, \quad n^m \in O(n^p) $$
  \item polinomi:
    $$ p(x) \in R[x], \quad deg(p) = m, \quad p(x) \in O(n^m) $$
\end{itemize}

si nota che esistono funzioni tra di loro incommensurabili. Inoltre, per qualsiasi $k$:
$$ \forall k, \quad n^k \in O(a^n), \quad \forall n > 1 $$
ovvero qualsiasi polinomiale è sempre migliore di ogni esponenziale.
\par
Per continuare la discussione della complessità degli algoritmi, occorre poi definire anche la: \\

\textbf{Notazione $\Omega$ grande} \\
La notazione $\Omega$ grande definisce effettivamente l'opposto della O grande:
$$ f(n) \in \Omega(g(n)) \Leftrightarrow \exists n_0, \quad c > 0 \quad \mathrm{t.c.} \forall n > n_0 : f(n) \geq cg(n) $$

\textbf{Notazione $\Theta$ grande} \\
La notazione $\Theta$ grande correla due funzioni che condividono lo stesso ordine di complessità:
$$ f(n) \in \Theta(g(n)) \Leftrightarrow f(n) \in O(g(n)) \land f(n) \in \Omega(g(n)) $$
oppure ancora:
$$ f(n) \in \Theta(g(n)) \Leftrightarrow \exists n_0 \quad \exists c_1,c_2 \quad \mathrm{t.c.} \quad \forall n \geq n_0 :
c_1g(n) \leq f(n) \leq c_2g(n) $$

per la notazione $\Omega$ e $\Theta$ grande esistono inoltre le regole:

\begin{itemize}
  \item antisimmetria O grande / $\Omega$ grande:
    $$ f(n) \in O(g(n)) \Leftrightarrow g(n) \in \Omega(f(n)) $$
  \item simmetria $\Theta$ grande:
    $$ f(n) \in \Theta(g(n)) \Leftrightarrow g(n) \in \Theta(f(n)) $$
  \item per $\Theta$ grande e $\Omega$ grande valgono le regole definite per O grande, ovvero dei
    fattori costanti, somma e prodotto, transtività, potenze e polinomi.
    Si nota soratutto che per ogni polinomio: $$ p(x) \in R[x], \quad deg(p) = m, \quad p(x) \in \Theta(n) $$
\end{itemize}

in una nota un'attimo più formale, si può affermare che le notazioni O grande, $\Omega$ grande e $\Theta$ grande
stabiliscono sull'insieme delle funzioni naturali $f(n)$ delle classi di equivalenza, rispettando effettivamente
le tre proprietà di riflessività, simmetria e transitività. Inoltre, si può dire che l'appartenenza ad una classe
di complessità stabilisce fra funzioni una relazione d'ordinamento, nell'ordine:

$$ O(1) < O(\log{n}) < O(n) < O(n\log{n}) < O(n^m) < O(m^n) < O(n^n) $$

Si ricorda inoltre che queste notazioni considerano solamente l'andamento \textit{asintotico} di una funzione,
in quanto tutto ciò che accade prima di $n_0$ è effettivamente irrilevante all'analisi che vogliamo fare (almeno
dal punto di vista della scalabilità).

\section{Complessità degli algoritmi iterativi}
Si presenta ora un semplice modello di calcolo per programmi scritti in pseudolinguaggio C++, compatibile
con i tre paradigmi della programmazione strutturata (esecuzione sequenziale, condizionale ed iterativa).
\par
Quello che vogliamo definire è una funzione del tipo:
\begin{center}
  C[costrutti del linguaggio] $\rightarrow$ classi di complessità
\end{center}
useremo inoltre le abbreviazioni:
\begin{center}
  V: costanti, I: variabili, E: espressioni, C: comandi
\end{center}

iniziamo con semplici operazioni di assegnamento e lettura variabili, valutazioni di espressioni, selezione
da vettori nonché l'istruzione di salto \textit{return}:
$$ C[V] = C[I] = O(1) $$
$$ C[E_1 \circ E_2] = C[E_1] + C[E_2] $$
$$ C[I[\:]] = C[E] $$
$$ C[I = E] = C[E] $$
$$ C[I[E_1] = E_2] = C[E_1] + C[E_2] $$
$$ C[\mathrm{return} \ E] = C[E] $$

si definisce poi per l'esecuzione sequenziale:
$$ C[{C_1,...,C_n}] = C[C_1] + ... + C[C_n] $$ 

per l'esecuzione condizionale:
$$ C[\mathrm{if}(E)C] = C[E] + C[C] $$
$$ C[\mathrm{if}(E) \: C_1 \: \mathrm{else} \: C_2] = C[E] + \max(C[C_1], C[C_2]) $$ 

e per l'esecuzione iterativa ($O(g(n))$ è il numero di iterazioni in funzione della dimensione d'istanza):
$$ C[\mathrm{for}(E_1, E_2, E_3)C] = C[E_1] + C[E_2] + (C[C] + C[E_2] + C[E_3]) \cdot O(g(n)) $$
$$ C[\mathrm{while}(E)C] = C[E] + (C[C] + C[E]) \cdot O(g(n)) $$

si definisce inoltre per la chiamata di funzione:
$$ C[F(E_1,...,E_n)] = C[E_1] + C[E_n] + C[{C...C}] $$

\section{Caratterizzazione di caso peggiore, migliore e medio}
Definiamo matematicamente le nozioni di caso peggiore, migliore e medio sull'insieme di istanze $In$:
\begin{itemize}
  \item Caso peggiore:
    $$ T_{worst} = max_{In}(\mathrm{tempo}(I)) $$
  \item Caso migliore:
    $$ T_{best}  = min_{In}(\mathrm{tempo}(I)) $$
  \item Caso medio (la funzione $P(I)$ restituisce la probabilita di verificarsi di una certa istanza):
    $$ T_{average} = \sum_{In}(\mathrm{tempo}(I) \cdot P(I)), \quad P(I) \in [0, 1] $$
    da cui possiamo ovviamente dire:
    $$ T_{worst} \leq T_{average} \leq T_{best} $$
\end{itemize}

\section{Complessità di algoritmi di uso comune}
Iniziamo la discussione di alcuni algoritmi di ordinamento di vettori in loco (in-place), basati sul confronto,
ammettendo l'accesso diretto in memoria in tempo $O(1)$. Definiamo innanzitutto una funzione di scambio con
ausilio di variabile temporanea:
\begin{lstlisting}[language=C++]
void swap(int& a, int& b) {
  int temp = a;
  a = b;
  b = temp;
}
\end{lstlisting}
\par \medskip
\textbf{Selection sort} \\
Il selection sort lavora in maniera incrementale su n elementi, selezionando iterativamente l'i-esimo elemento
e confrontandolo con glin-i elementi successivi in modo da scambiarlo con il minimo trovato. Dopo n-1 
iterazioni il vettore risultaordinato. In pseudocodice:

\begin{lstlisting}[language=C++]
void selectionSort(int vett[], int n) {
  for(int i = 0; i < n - 1; i++) {
    int m = i;
    for(int j = i + 1; j < n; j++) {
      if(vett[j] < vett[m]) m = j;
    }
    scambia[vett[m], vett[i]];
  }
}
\end{lstlisting}

il selection sort presenta caratteristiche identiche nel suo caso migliore e peggiore, effettuando sempre le stesse
operazioni su qualsiasi tipo di istanza. In particolare, la complessità dipende dal numero di confronti:
$$ (n-1)+(n-2)+(n-3)+...+2+1 = \sum_{i=1}^{n} i = \frac{n(n-1)}{2} \in O(n^2) $$
e il numero di scambi, che è $O(n)$. Alternativa al selection sort può essere il
\par \medskip
\textbf{Bubble sort} \\
Il bubble sort (algoritmo di ordinamento a bolle) non adotta un'approccio incrementale, ma si limita a scambiare
da destra verso sinistra le coppie adiacenti non ordinate n-1 volte, finchè il vettore non risulta ordinato.
In pseudocodice:

\begin{lstlisting}[language=C++]
void bubbleSort(int vett[], int n) {
  for(int i = 0; i < n - 1; i++) {
    for(int j = n-1; j > i; j--) {
      if(vett[j] < vett[j - 1]) scambia(vett[j], vett[j - 1]);
    }
  }
}
\end{lstlisting}

a differenza del selection sort, il bubble sort un numero sia di confronti che di scambi pari a $O(n^2)$.
Questo apparente difetto è però compensato da una possibile ottimizzazione disponibile in fase di scorrimento del
vettore: basterà infatti controllare eventuali scorrimenti senza scambi per poter ottenere una complessità
temporale in caso migliore di $O(n)$ (caso di scorrimento singolo al primo ciclo). In pseudocodice:

\begin{lstlisting}[language=C++]
void betterBubbleSort(int vett[], int n) {
  for(int i = 0; i < n - 1; i++) {
    bool q = false;
    for(int j = n-1; j > i; j--) {
      if(vett[j] < vett[j - 1]) {
        scambia(vett[j], vett[j - 1]);
        q = true;
      }
    }
    if(q == false) {
      break;
    }
  }
}
\end{lstlisting}

si può in generale applicare i limiti asintotici sia al costo di un singolo algoritmo che alla complessità di 
un intero programma. Possiamo ad esempio definire, su un qualsiasi problema:
\begin{itemize}
  \item Limite asintotico superiore:
    il minimo ordine di complessità superiore a tutti gli ordini di complessità degli algoritmi atti a risovlere
    il problema (in sostanza caso peggiore)
  \item Limite asintotico inferiore
    il massimo ordine di complessita inferiore a tutti gli ordini di complessità degli algoritmi atti a risolvere
    il problema, quindi il miglior caso possibile per un qualsiasi algoritmo. \par
    Ad esempio, per quanto riguarda l'ordinamento di vettori monodimensionali, il limite asintotico inferiore
    è pari a $O(n\log{n})$.
\end{itemize}

\textbf{Ricerca lineare}
L'algoritmo di ricerca lineare di una certa chiave $x$ su un certo vettore scansiona ogni singolo elemento confrontandolo
alla suddetta, ottenendo complessità (appunto) lineare $O(n)$.
In pseudocodice:
\begin{lstlisting}[language=C++]
int linearSearch(int vett[], int n, int x) {
  int b = 0;
  for(int i = 1; i < n; i++) {
    if(vett[i] == x) b = 1;
    return b;
  }
}
\end{lstlisting}
si nota che il caso migliore (quello in cui il primo elemento è quello cercato) ha complessità $O(1)$.
\par \medskip

\textbf{Divisioni ripetute} \\
Si riporta un algoritmo che divide un certo numero per 2 ciclicamente fino ad azzerarlo (ovviamente sul gruppo degli
interi), magari per volerlo convertire in base 2:
\begin{lstlisting}[language=C++]
int div_2(int n) {
   int i = 0;
   while(n > i) {
   n = \frac{n}{2}; i++;
   }
   return i;
}
\end{lstlisting}

possiamo quindi estrapolare una misura della complessità sul caso peggiore (quello in cui $n$ è potenza di 2):

$$ n = 2^m \Leftrightarrow m = \log_2{n} \rightarrow O(log(n)) $$

\textbf{Ricera binaria} \\
L'algoritmo di ricerca binaria (o ricerca dicotomica) effettua la stessa operazione della ricerca lineare, ma
parte dall'assunto che il vettore sia già stato ordinato per approfittare di una notevole ottimizzazione, che porta la complessità
temporale a $O(log(n))$. L'ottimizzazione usata si basa sul confronto della chiave con l'elemento centrale del vettore
(e non di ogni suo elemento), che permette di dimezzare ciclicamente il vettore ottenendo quindi una riduzione
a velocità logaritmica della dimensione d'istanza. In pseudocodice l'implementazione ricorsiva:

\begin{lstlisting}[language=C++]
int binarySearch(int vett, int min, int max, int x) {
  if(max < min) return -1;
  int med = (min + max) / 2;
  if(vett[med] == key) {
    return med;
  }
  if(vett[med] > key) {
    return binarySearch(0, med - 1, key, dest);
  } else {
    return binarySearch(med + 1, max, key, dest);
  }
}
\end{lstlisting}

\end{document}
